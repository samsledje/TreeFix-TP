\title{TreeFixVP}
\author{
        Mukul Bansal \\
                Department of Computer Science\\
        University of Connecticut\\
            \and
        Samuel Sledzieski\\
            Department of Computer Science\\
        University of Connecticut\\
}
\date{\today}

\documentclass[12pt]{article}

\begin{document}
\maketitle

\section{Introduction}
TreeFixVP is a program for reconstructing highly accurate viral phylogenetic trees.
TreeFixVP incorporates host information with sequence data to reconstruct the tree using 
a mixed maximum likelihood / maximum parsimony framework. Given a maximum likelihood phylogeny
and a multiple sequence alignment, TreeFixVP searches among multiple topologies which are supported
equally by sequence data (using the Shimodaiara-Hasegawa test statistic). The selected topology
is the one that minimizes then umber of necessary transmissions.

\paragraph{Requirements}
    \begin{itemize}
        \item Python (2.5.4 or greater)
        \item C compiler (gcc)
        \item SWIG (1.3.29 or greater)
        \item Numpy (1.5.1 or greater)
        \item Scipy (0.7.1 or greater)
        \item Additionally, Python modules are required for computing the p-value for likelihood equivalence.
    \end{itemize}

\paragraph{Likelihood}
TreeFixVP uses the Shimodaira-Hasegawa (SH) test statistic with RAxML site-wise likelihoods to compute
p-values for each candidate tree.

\paragraph{Parsimony}
TreeFixVP scores each statistically equivalent candidate tree using Fitch's algorithm.
Given host labels at the leaf nodes, the Fitch module computes a score equivalent to the
minimum necessary number of transmissions needed to label the internal nodes.

\section{Tutorial}
\paragraph{Input}
TreefixVP requires a seed tree, generally a maximum likelihood tree, and a multiple sequence alignment

\subsection{Usage}
treefixVP -[options] \textless treefile\textgreater

\paragraph{Options}
TreeFixVP assumes that the multiple sequence alignment and seed tree have the same root name, with
different extensions. The default extensions are ".align" and ".tree", but other extensions can be specified.
The output tree file will have the same root name, and will have either the default extension
".treefix.tree", or a user specificed extension.

\begin{itemize}
    \item -A \textless alignment file extension\textgreater, -{}-alignext=\textless alignment file extension\textgreater
    \item -o \textless old tree file extension\textgreater, -{}-oldext=\textless old tree file extension\textgreater
    \item -n \textless new tree file extension\textgreater, -{}-newext=\textless new tree file extension\textgreater
\end{itemize}

The default likelihood model uses RAxML site-wise likelihoods, but a different user-defined
module can be substituted. The default likelihood test is the SH test, but again a user-defined
test can be used. 

\begin{itemize}
    \item -m \textless module for likelihood calculations\textgreater, -{}-module=\textless module for likelihood calculations\textgreater
    \item -e \textless extra arguments to module\textgreater, -{}-extra=\textless extra arguments to module\textgreater
    \item -t \textless test statistic\textgreater, -{}-test=\textless test statistic\textgreater
    \item -{}-alpha=\textless alpha\textgreater $\rightarrow$ alpha threshold (default: 0.05)
    \item -p \textless alpha\textgreater, -{}-pval=\textless alpha\textgreater
\end{itemize}

The default module for transmission cost uses Fitch's algorithm. A user defined cost model can
be substitued.

\begin{itemize}
    \item -M \textless module for transmission cost calculation\textgreater, -{}-smodule=\textless module for transmission cost calculation\textgreater
\end{itemize}

A number of options can be specified to guide the search conducted by TreeFixVP.

\begin{itemize}
    \item -x \textless seed\textgreater, -{}-seed=\textless seed\textgreater $\rightarrow$ seed value for random generator
    \item -{}-niter=\textless number of iterations\textgreater $\rightarrow$ number of iterations (default: 1000)
    \item -{}-nquickiter=\textless number of quick iterations\textgreater $\rightarrow$ number of subproposals (default: 100)
\end{itemize}

Finally, there are a handful of informational options.

\begin{itemize}
    \item -{}-version $\rightarrow$ show program's version number and exit
    \item -h, -{}-help $\rightarrow$ show the help message and exit
    \item -V \textless verbosity level\textgreater, -{}-verbose=\textless verbosity level\textgreater
    \item -l \textless log file\textgreater, -{}-log=\textless log file\textgreater $\rightarrow$ log filename.  Use '-' to display on stdout.
\end{itemize}

\section{Test Results}
TreeFixVP was tested on 8 sample trees and multiple sequence alignments. Below is a list of
the 8 datasets. The column labeled "Old Score" is the Fitch score of the original tree, and the column
labeled "New Score" is the Fitch score of the tree reconstructed by TreeFixVP. Also included are the number
of leaves for the given tree, and the time of the run.

\vspace{5mm}
\begin{tabular}{ |c|c|c|c|c| }
    \hline
    \textbf{Name} & \textbf{Leaves} & \textbf{Old Score} & \textbf{New Score} & \textbf{Runtime (minutes)} \\
    \hline
    AA & 118 & 7 & 5 & 48 \\
    \hline
    AB & 48 & 2 & 1 & 28 \\
    \hline
    AC & 86 & 3 & 3 & 32 \\
    \hline
    AD & 76 & 2 & 1 & 67 \\
    \hline
    AE & 29 & 3 & 2 & 15 \\
    \hline
    AH & 53 & 2 & 1 & 31 \\
    \hline
    AI & 163 & 71 & 51 & 71 \\
    \hline
    AJ & 49 & 3 & 3 & 21 \\
    \hline
\end{tabular}

\subsection{Example}
\textbf{Find a sample dataset that we can release and walk through
an example here}

\section{Changes from TreeFixDTL}

\subsection{treefixVP}
\begin{itemize}
    \item Renamed treefixDTL executable
    \item Removed most options in parser, only need alignment extension, old extension, new extension
    \item Changed default smodule to FitchModel
    \item Call treefixVP rather than treefix
\end{itemize}

\subsection{fitchmodel.py}
\begin{itemize}
    \item Copy of rangerdtlmodel.py
    \item Changed compute{\_}cost() to call fitch.linux executable
    \item Removed stree, smap from FitchModel.optimize{\_}model
    \item Removed call to CostModel.optimize{\_}model()
\end{itemize}
\subsection{treefix{\_}for{\_}VP}
\begin{itemize}
    \item Renamed treefix to treefix{\_}for{\_}VP - this avoids path collision with treefix and treefixDTL
    \item Removed smap and stree as required arguments
    \item Changed to common.check{\_}req{\_}options(species=False)
    \item Removed stree and gene2species from check{\_}input{\_}tree()
    \item Removed stree and gene2species from search{\_}landscape()
    \item Removed check{\_}congruent{\_}tree() function, no longer applies
    \item Removed 'if flag: return mintree'
    \item Removed reading species tree and species map from main()
    \item Removed stree and gene2species from calls to optimize{\_}model(), check{\_}input{\_}tree(), and search{\_}landscape() in main
\end{itemize}

\subsection{General}
\begin{itemize}
    \item Removed all ranger executables
    \item Removed treefix{\_}compute
    \item Removed rangerdtl model from models module
    \item Modified setup files to install the proper executables
    \item Updated treefixVP.py with version and software info
    \item Updated INSTALL.txt, CHANGES.txt, and README.txt
    \item Added README.pdf
\end{itemize}

\end{document}